\subsection{Problem}
Many systems are traditionally designed by envisioning a single person sitting at their desktop or laptop computer, but in practice, we know this is not the way many systems are used. In certain circumstances, for example, an ambulance or emergency room, systems are simply workstations for whoever needs them. The workspaces don't have dedicated users like we think of in a typical office setting.

\subsection{Proposed Solution}
Distributed Cognition is a framework that aims to solve this problem by considering the bigger picture. Specifically, it aims to solve design problems, but the idea proposed by the authors of this paper is that Distributed Cognition may also apply to the distribution of labor. This is what is being coined as `DiCoT' (Distributed Cognition for Teamwork).

This was attempted in a case study using the London Ambulance Service (LAS). Data was gathered through various different roles in the organization and analyzed using DiCoT. The core principles of DC were applied to the data gathered and two alternative redesigns of the system were proposed.

\subsection{Validation}
Because the proposed methods were not implemented at the time the paper was written, there was no validation done to determine the effectiveness of the proposed modifications. Instead the author proposes that it is simply a `proof of concept.''

\subsection{Limitations}
The largest limitation of the study was the lack of any concrete proof that the methods were successful. Although the ideas generated using the DiCoT framework look promising, there is no actual implementation to gauge whether or not they improved the workflow of a team.
