\begin{table}%
\caption{Synthesis Matrix}\label{tab:synthesis-matrix}
\begin{minipage}{\columnwidth}
\begin{center}
\begin{tabular}{|p{0.75in}|p{1.5in}|p{1.5in}|p{1.5in}|}
\hline
Idea & Hutchins\cite{hutchins2000distributed} & Blandford, et al.\cite{blandford2005dicot} & Buchan, et al.\cite{buchan2020applying} \\
\hline
Distributed Cognition as Cognitive Science &
Originally proposed the new branch of science in an original paper. &
Absolutely they see value in the application of this science, as they have created a framework to apply the principles of DC to Teams. &
Yes, as they have applied DiCoT to Agile Requirements Prioritization.
\\
\hline
Distributed Cognition as applicable to Teams &
Not directly stated but strongly implied due to the nature of DC itself. &
Blandford, et al.\ focused largely on the application of Distributed Cognition to Teams and created the DiCoT framework. &
Buchan, et al.\ worked directly with a software development team to apply the principles of the DiCoT framework during their analysis.
\\
\hline
Distributed Cognition in regard to Agile Processes &
Agile was not specifically mentioned, as the original paper where DC was proposed as a branch of science took a very high-level view of cognition as it relates to groups. &
Agile was not called out in the work done by the DiCoT researchers, as they were focused on the more generic case of simply ``teams,'' but the principles definitely apply. &
Yes, the conclusion of this research was that DC, specifically applied through the DiCoT framework's principles, is both applicable and beneficial in an Agile context.
\\
\hline
\end{tabular}
\end{center}
\bigskip\centering
\end{minipage}
\end{table}%
