The beginnings of the field of Distributed Cognition hinged largely on "gut feeling" rather than material evidence. The confusing results of initial attempts to quantify and define what principles govern how these processes work and can be utilized to improve real-world processes created a rocky start for the application of DC. However, due to the continued research in the field, we are beginning to see how the dots connect and can improve real-world scenarios.

Although the results of the London Ambulance research were not able to be tested, we can clearly map productive Agile requirements prioritization processes to the principles defined in the Distributed Cognition for Teams framework. This shows a promising future for the field of DC with respect to team interactions. Not only can this framework be used to produce new effective team processes, it can be used to analyze and improve existing ones.

In answering whether or not a simple framework could exist that could be applied to all Agile requirements prioritization meetings to prioritize backlog items, the answer is that we simply don't know. We have research that has identified six distinct criteria that was used to determine the value of a backlog item, but more research is needed to determine whether or not this is a complete set (and certainly whether it can be applied in a general fashion). What we do know, is that these meetings rely heavily on the participants having a shared understanding of the different aspects discussed in the meeting. This takes time to develop simply through conversation and discussion of the various topics being discussed. However, we can safely draw the conclusion that it is likely helpful and productive to utilize the DiCoT principles as a generic checklist to ensure that these meetings are being run as productively as they can be --- or at the very least, identify areas for improvement.
