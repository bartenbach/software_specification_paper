\documentclass[acmsmall,screen]{acmart}

\begin{document}

\title{Applying Distributed Cognition in Teams to Agile Processes}

\author{Blake Bartenbach}
\email{bbartenbach@unomaha.edu}
\affiliation{%
  \institution{University of Nebraska at Omaha}
  \address{6001 Dodge Street}
  \city{Omaha}
  \state{Nebraska}
  \country{USA}
  \postcode{68198}
}
\date{October 2021}

\acmBooktitle{}

\begin{abstract}
Placeholder
\end{abstract}

\keywords{agile, dicot, dc, distributed cognition, prioritization, scrum}

\maketitle

\section{Introduction}
Prioritizing customer requirements and communicating the value of stories with stakeholders is not an uncommon challenge to anyone who has served as a product owner, proxy product owner, or even a Scrum master of a Scrum team. The attendees of a backlog refinement and/or sprint planning session can have vastly different perspectives, knowledge, backgrounds, and experience --- not to mention individual stakes in the work being performed. Without a common framework for prioritization, this makes prioritizing stories and tasks challenging as the relative requirements of tasks are subjective. This is the problem Buchan, Zowghi, and Bano\citation{buchan2020applying} set out to solve.
In a theory, the customer or designated stakeholder could be deferred to to resolve every single one of these conflicts. In practice, however, the customer is rarely available to attend every single Scrum backlog refinement session. Furthermore, the customer (nor any single individual) is unlikely to possess all of the technical and domain knowledge required to actually make an informed decision to deconflict the priorities.
For example, if a database team attends a backlog refinement meeting, they may be inclined to view the stories pertaining to the database as the highest priorities, as that is the component that they interface with on a daily basis. However, a network team may feel strongly that the network tasks are a higher priority. Management can attend without the technical insight and background to understand either of these tasks, and request that a metrics reporting task takes precedence. This example illustrates a common example of the challenges of a refinement session
when working within service teams.
Applying the principles of DC (Distributed Cognition), the DiCoT framework, and the knowledge gleaned from previous experiments using the DiCoT framework, ``Can the criteria identified by DiCoT analysis be distilled down to a (relatively) simple set of rules for prioritizing product backlog items?'' The aim of this research is to identify such a set, if one exists, or at the very least determine the feasibility of identifying such a set.

\section{Background}
Placeholder

\section{Summaries}
\subsection{}
\subsection{}
\subsection{}

\section{Discussion}
Placeholder


\section{Conclusion}
Placeholder


\section{References}
Placeholder

\bibliographystyle{ACM-Reference-Format}
\bibliography{references}

\end{document}
\endinput
