\subsection{Problem}
In this paper\cite{buchan2020applying}, the authors are trying to solve the problem of story backlog prioritization with stakeholders that can commonly occur in Agile settings.

Prioritizing requirements and communicating the value of stories with stakeholders is not an uncommon challenge to anyone who has served as a product owner, proxy product owner, or even scrum master. The attendees of a backlog refinement and/or sprint planning session can have vastly different perspectives, knowledge, backgrounds, and experience --- not to mention individual stakes in the work being performed. This can create a challenge without a common framework for prioritization as the relative requirements of tasks can be vastly different depending on who you ask.

If the database team attends the refinement, they may tend to view the story that benefits the database as the highest priority, as it is what they are most familiar with and interface with regularly. However, the network team may feel strongly that the network tasks are a higher priority. Management can attend without the technical insight and background to understand either of these tasks, and request that a metrics reporting task takes precedence. This example illustrates a common example of the challenges of a refinement session when working within service teams.

\subsection{Proposed Solution}
To begin to solve the problem, the authors sought to determine two important things:
1. What aspects of Agile Requirement Prioritization are cognitively significant?
2. What principles from the DiCoT framework are important in the prioritization process?
One solution to this problem is to utilize a common prioritization criteria. In the paper being discussed\cite{buchan2020applying}, a criteria was developed by applying DiCoT\cite{blandford2005dicot}: a methodology for applying Distributed Cognition (DC)\cite{hutchins2000distributed}.

The analysis identified six criteria for prioritization and three areas of distinct cognitive effort during a field study of two backlog refinement sessions for an undisclosed product.

\subsection{Validation}
There were some potentially identified threats to the validity of the study. Selection bias can not be ruled out, as the project studied was chosen by a single contact. External validity is also low, in the words of the authors, and applicability to other projects is likely to be inconsistent. Observer bias was also a possibility, as the team studied was aware that they were being analyzed during their backlog refinement meeting and may have behaved differently. The observer did attempt to mitigate the impact of this, however, by building a repertoire with the team prior to commencing the study.

\subsection{Limitations}
One limitation conceded to by the authors of the paper was that the process of applying the DiCoT framework and collecting data for analysis was very time consuming, and likely not practical to be conducted regularly.

\subsection{Research Question}
Can the criteria identified by this analysis be abstracted into a common framework? Is there some commonality between the requirements of all projects that can be quantified into a simple set of rules for prioritizing backlog items? The aim of this research is to attempt to identify such a set.
