\begin{table}%
\caption{Synthesis Matrix}
\label{tab:synthesis-matrix}
\begin{minipage}{\columnwidth}
\begin{center}
\begin{tabular}{|p{0.75in}|p{1.75in}|p{1.75in}|p{1.75in}|}
\hline
Hutchins\cite{cornelsen} & Stewart\cite{stewart} & Bruley, et al.\cite{bruley} \\
\hline
Alteration of women's roles because of WWII &
- Women accredited the WASP program for opening new doors, challenging stereotypes, and proving that women were as capable as men (p. 113)\newline
- Women could compete with men as equals in the sky because of their exemplary
performance (p. 116)\newline
- WASP created opportunities for women that had never previously existed (p. 112)\newline
- Women's success at flying aircrafts ``marked a pivotal step towards breaking the
existing gender barrier'' (p. 112) &
- WAAC (Women's Army Auxiliary Corp) was 1st chance for women to serve in army, given full army status in 1943 as WAC (p. 28)\newline
- Needs of the war were so great that women's traditional social roles were ignored (p. 30)\newline
- Military women paid well for the time period and given benefits if they became pregnant (p. 32)\newline
- The 1940's brought more opportunities to women than ever before (p. 26) &
- Women given equal opportunities (p. 223)\newline
- Women joined workforce as a break from the ordinary to help the war (p. 220)\newline
- Unconscious decision to cross into male-dominated roles (p. 221)\newline
- Seized these new opportunities to bring about change (p. 230) \\
\hline
Hardships and oppositions women faced &
- ``From the outset male pilots resented women's presence in a traditionally male military setting'' (p. 1113-4)\newline
- ``The WASP were routinely assigned inferior planes that were later found to have been improperly maintained'' (p. 114)\newline
- discrimination against WASP at every level of military service, women were only paid 2/3 of what men were for doing identical tasks (p. 114) &
- Women in the military given extensive physical and mental tests, but still discriminated against, ridiculed, and considered inferior to men (p. 29) &
- Women given unskilled labor positions by government because only seen as temporary workers, therefore no reason to train them (p. 221-2)\newline
- Women given less significant work and viewed as less intelligent and physically able (p. 224)\newline
- ``The Church-Bliss diary reveals how dilution arrangements \dots ensured that women working in male preserves were prevented from achieving any sort of equality'' (p. 230) \\
\hline
\end{tabular}
\end{center}
\bigskip\centering
\footnotesize\emph{Source:} Excerpt copied from {\url https://tutorial.dasa.ncsu.edu/wp-content/uploads/sites/29/2015/06/synthesis-matrix.pdf}
\end{minipage}
\end{table}%
