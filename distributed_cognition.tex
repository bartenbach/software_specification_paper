\subsubsection{Problem}
Distributed Cognition (DC) theory is a branch of cognitive science that aims to understand cognition beyond the basic sense in which it is usually viewed. Normally, we view the concept of cognition as something that is limited to a single individual's ``skin or skull'' as the paper puts it. DC aims to understand cognition that is distributed among individuals in groups. More specifically, the goal of DC in Hutchins' paper was to identify how tasks that are commonly only done by a single individual could be distributed to a group.

\subsubsection{Proposed Solution}
In his research, Hutchins did not identify a solution, but rather explained the concepts as he identified them, and proposed some interesting questions that warranted further investigation.
