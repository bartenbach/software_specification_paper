\subsection{The Basis of Distributed Cognition}
To begin to decide whether or not it is possible to extract a simple, repeatable formula to assist Agile teams with prioritization, we must first examine the credibility of the research done so far. Starting with the original paper where Distributed Cognition was first proposed by Edwin Hutchins in the 1990s, we must first ask ourselves an important fundamental question that sets the framework on which everything that follows will be built: Is Distributed Cognition simply a theory, or is it something concrete than can be quantified, defined, and clearly demonstrated?

In the opening section of the paper, Hutchins compares two published works, Mind in Society (Vygotsky, 1978) and Society of Mind (Minsky, 1985). Hutchins suggests that the titles of books are too similar to be merely a coincidence and that something larger might be happening in ``systems of distributed processing.''\cite{hutchins2000distributed} This initial suggestion sets the tone for the rest of the paper quite well, as it is merely an unsubstantiated opinion that something could be at play. The issue here, is that Distributed Cognition has been described as an approach to cognitive science but this suggestion (as well as many of the others that follow) are not scientific.

In a paper referenced by Hutchins, Halbwachs (1925) declares that it doesn't make sense to discuss memory as a property of a single individual. Although this does pose some interesting questions and perhaps warrants further research, memory can certainly be discussed in the context of a single individual. The famous case of British musician Clive Wearing illustrates this counterpoint well. Wearing contracted herpesviral encephalitis and developed both anterograde and retrograde amnesia; that is, Wearing is both unable to form new memories, and unable to recall past memories. This is a case in where it certainly seems to make sense to discuss a single individual's memory.

Looking at this from a broader perspective however, it could also be argued that while it does make sense to discuss Wearing's memory as an individual, Wearing can also be viewed as a phenomenal example of Distributed Cognition at work. While Wearing's own memory has been rendered almost completely useless to him, he is able to rely on his social network to support him and provide him the memories he can no longer access himself. A good example of this is that Wearing's wife, children, and caregivers can direct him to the piano where he can play complex pieces and sight read. Although Wearing has no episodic memory, his procedural memory is still intact, and he was able to conduct a choir. Of course, these things were only possible because Clive was able to (unintentionally) leverage his social network to provide him proverbial paths to his lost skills and memories. This appears to demonstrate distributed cognition among these individuals. In specifically Clive's case, the cognition of others is required for many meaningful cognition to occur.

At another point, Hutchins states that the environment that humans are thinking in are not natural environments, but ``artificial through and through.''\cite{hutchins2000distributed} We again must view this through a critical lens and ask ourselves if this is truly a scientific fact. If human beings are living in an environment created by them without outside assistance, is that not a natural environment? Is the distinction between a natural environment and an artificial environment simply that humans have built their environment? If so, this is a meaningless distinction as birds build nests, beavers build dams, and ants build very complex systems of underground tunnels. Hutchins continues on to say that, ``It does not seem possible to account for the cognitive accomplishments of our species by reference to what is inside our heads alone.''\cite{hutchins2000distributed} Whether or not it seems possible is scientifically irrelevant --- what is important is whether or not it is factually true.

Although Hutchins proposes some interesting theories, we highlight that at the time of the original paper, many of the ideas proposed and claims made had no scientific references or field studies to support their validity.

\subsection{Applicability to Teams}
When we examine the research in the previous paper, it suggests many interesting ideas and areas of further research, but there is a lack of evidence to support the theories. Regardless, the idea of distributing cognitive processes across teams remains an interesting concept to be explored further.

The DiCoT framework opens with the claim that it has, ``been developed and tested within a large, busy ambulance control centre.''\cite{blandford2005dicot} Initially, this is an impressive statement and suggests that the framework has been shown to be effective in a field test. However, upon examination of the methods used in the study, it is revealed that what this is referring to is two half-day visits to the London Ambulance Service creating a proof-of-concept that had not been tested. Blandford and Furniss applied DiCoT using the information gathered but stated, `Since these re-designs have not been implemented, it is not possible to validate the conclusions drawn, but this exercise has been a `proof of concept'\dots'\cite{blandford2005dicot}

While this does not prove that the DiCoT framework does not work, it also does not provide any supporting evidence that the kind of analysis and artifacts created during the research can provide any real-world benefit. Regardless of the lack of tangible evidence of the method's success, many of the principles that the method was proposes logically and anecdotally make sense. The framework provides 18 principles that should be followed when designing any team process with distributed cognition in mind. These are abstract guidelines that can be utilized to (purportedly) enhance the effectiveness of a team. An example provided by the DiCoT paper is `creating reminders of where a task was left off at', which follows the principle, `Creating a scaffolding to simplify cognitive tasks.` Although this is clearly a guideline for cognition in the general sense, it is not clear how this particular principle applies to Distributed Cognition specifically.

\subsection{Applicability to Agile Prioritization}
When we look at the theory and framework from a scientific standpoint based upon the evidence we have thus far, one may draw the conclusion that there is no valid scientific evidence to warrant further consideration. However, interestingly enough what we find is that this is where everything seems to come together.

Albeit the research examined previously does not factually demonstrate anything aside from a theory, perhaps the greatest benefit for ``connecting the dots'' is examining a process that is known to be cognitively demanding, and looking at the best practice processes and procedures around it

Initially, it becomes obvious that simply attending two requirements prioritization meetings is not enough research to make any definitive claims about the effectiveness of applying the DiCoT framework principles to the Agile requirements engineering process. However, the research is very promising, and certain warrants more investigation. The research team applied all 18 principles of the DiCoT framework during their analysis, and uncovered very promising insight into why these meetings can be effective, and perhaps some pitfalls that can produce ineffective requirements engineering meetings.

One issue we can immediately identify from this field study, is that it was done very early in the requirements engineering process --- before development had even started. The reason this particular information could cause concern is that this phase of pre-development requirements engineering is brief, certainly compared to the amount of time that the software is actively developed and sustained. Some factors that make this relevant to point out is that the development team and software testers were not present in the meetings attended by the researchers. Moreover, these roles being present provide different perspectives into the prioritization process and they were not accounted for in this study. The dynamics between meeting attendees will change over time as they develop a shared understanding of the priorities, and new techniques may need to be explored to handle the later stages of requirements engineering. The researches do acknowledge this, however, and agree that further research is needed in the area.

The researchers identified several relevant cognitive artifacts: user story cards that were physically arranged by the attendees, a whiteboard, and a projected computer screen. These elements are certainly present in many Agile requirements prioritization meetings, but how important are they?

What the team found, is that the meeting tended to consist of a repeating process: an unprioritized user story is taken from a group of unprioritized items, and the attendees would discuss and eventually identify the position of the element in relation to the already prioritized elements. The team also identified the cognitively demanding aspects of the process:
\begin{enumerate}
\item Explaining and reasoning about stakeholders' perspectives on the value of the story.
\item Agreeing on a position for the item in the prioritized backlog.
\item Reasoning, questioning, and clarifying the intent of user stories to develop a shared understanding of the value provided.\cite{buchan2020applying}
\end{enumerate}

The team further identified six prioritization criteria used to determine the position of a story in the backlog:
\begin{enumerate}
\item The strategic value of the story to the case organization.
\item The strategic value of the story to the end users.
\item The negative impact of not implementing the user story.
\item The cost and effort versus the benefit of implementing the story.
\item The negative impact on internal stakeholders with dependencies affected by the story.
\item The negative impact of dependencies between the particular story being examined and other user stories.\cite{buchan2020applying}
\end{enumerate}

Furthermore, the team was able to identify and highlight exactly which processes directly connected to principles identified in the DiCoT framework. The role of the user story cards alone highlighted many principles of the DiCoT framework. Not only did they function as information radiators, information buffers, information filters, information transformers, and attention coordinators, but they also allowed the facilitator to utilize them as behavior triggers to move the meeting along.\cite{buchan2020applying} Furthermore, they acted as visual indicators of the meeting progress (stories remaining to be prioritized). This connection with the DiCoT framework principles strongly suggests that these user story cards are an important part of the Agile requirement prioritization process.

The designated facilitator of the meeting was identified as an important element for the success of the meeting. A skilled facilitator will be able to identify when a consensus has been reached, and move the meeting along to the next backlog item to be discussed. However, the team cautions that this position of power can easily allow the facilitator of the meeting to exert more influence than others over discussion items.

The role of face-to-face contact (typically highlighted by the Scrum framework) was determined to be integral to enhancing the ease of communication in the meeting. The team identified a wide variety of body language used throughout the meetings such as hand gestures to provide emphasis, pointing to particular items, heads nodding or shaking in agreement/disagreement, and other nonverbal cues.

The application of the DiCoT framework and analysis performed also identified the meeting room to be an important piece to be considered. The cognitive artifacts used (the whiteboard, user story cards, and projected computer screen) were integral to the flow of information and recording decisions made during the meeting. The team also highlighted the importance of having a distraction-free area for the team to focus on the meeting.

Lastly, the diversity of the attendees was identified as an important aspect for conducting effective requirements prioritization. This was demonstrated by several events where a previously high priority story was discarded due to new information, and a previously low priority story became high priority for the same reason. Having a variety of perspectives allows the team to make much more informed decisions about priorities and the state of work yet to be performed.

The overall results of the research were promising. The connected well-known ceremonies to principles defined in the DiCoT framework, and uncovered the reasons why they are effective. The team proposes that the DiCoT principles could be used as a checklist that can be used to analyze the effectiveness of a current Agile requirement prioritization process, or as a guide that can be consulted if the meetings are not going well. The results of the assessment can be used to facilitate positive change in the process.
