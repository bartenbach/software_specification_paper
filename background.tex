Although originally geared towards software projects, many teams in today's industry have adapted Agile methodologies similar to Scrum as their project management framework. While Scrum does a great job of defining roles and ceremonies to empower teams to work more effectively, it does not prescribe the most effective way to run these ceremonies in detail. Although having any regular prioritization meeting is arguably better than none, in the interest of streamlining and improving this process some researchers have turned to Cognitive Science for answers. In this paper, we review these experiments with a skeptical lens to determine to scientific validity of applying this branch of science to a popular event in Scrum/Agile frameworks known as Agile Requirements Prioritization. In this meeting, at least from a software perspective, most often software developers, their scrum master, their product owner, and anywhere from zero to many stakeholders attend a time-boxed meeting that typically lasts an hour to discuss the priorities of the project. During this meeting decisions are made about which tasks and features need to be worked sooner, and which can be worked later. The goal of this meeting is to refine the product's backlog of work into a coherent, list of tasks that reflect the ordered priorities agreed upon by the stakeholders and development team of the product.
